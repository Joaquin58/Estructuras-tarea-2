\documentclass[oneside]{book}
\usepackage{epsfig,graphicx} % Required for inserting images
\usepackage{amsmath}
\usepackage{amsthm}
\usepackage{amssymb}
\usepackage{subcaption}
\usepackage[spanish,mexico]{babel}
\usepackage[bookmarksopen]{hyperref}
\usepackage[utf8]{inputenc}
\usepackage{array}
\usepackage{listings} %Soporte para código
\usepackage[left=2cm,right=2cm,top=1.8cm,bottom=2.3cm]{geometry}
\usepackage{multicol}
\usepackage{enumitem}
\usepackage{blindtext}
%\usepackage{schemata}
% ---definición de los paquetes--
\usepackage{fancyhdr}            % Permits header customization. See header section below.
\fancypagestyle{plain}{
\lhead{}
\fancyhead[R]{\thepage}
\fancyhead[L]{}
\renewcommand{\headrulewidth}{0pt}
\fancyfoot{}
}
\pagestyle{fancy}
\fancyhead[R]{\thepage}
\fancyhead[L]{}
\title{Tarea 02: Lógica Proposicional}
\author{Ramírez Mendoza Joaquín Rodrigo\\
Villalobos Juárez Gontran Eliut\\
Treviño Puebla Héctor Jerome}
\date{\today}
% ---Inicio de la portada
\begin{document}
\begin{titlepage}
	\begin{minipage}{3cm}
		\begin{center}
			\includegraphics[height = 0.14\textheight]{recursos/Logo_UNAM.png}\par
		\end{center}
	\end{minipage}\hfill
	\begin{minipage}{10cm}

	\end{minipage}\hfill
	\begin{minipage}{3cm}
		\begin{center}
			\includegraphics[height = 0.14\textheight]{recursos/Logo_FC.png}\par
		\end{center}
	\end{minipage}
	\centering
	\vspace{1cm}

	{\bfseries\LARGE Universidad Nacional Autónoma de México \par}

	\vspace{1cm}
	{\scshape\Large Facultad de Ciencias \par}
	\vspace{1cm}
	{\scshape\Large Estructuras Discretas \par}
	\vspace{1cm}
	{\scshape\Large Licenciatura en Ciencias de la Computación \par}
	\vspace{1cm}
	{\scshape\Huge Tarea 02: Lógica Proposicional.  \par}
	\vspace{3cm}
	{\itshape\Large Segundo Parcial \par}
	\vfill
	{\Large Autores: \par}
	{\Large Ramírez Mendoza Joaquín Rodrigo \par}
	{\Large Villalobos Juárez Gontran Eliut\par}
	{\Large Treviño Puebla Héctor Jerome \par}
	\vfill
	{\Large Octubre 2024 \par}
\end{titlepage}
% ---Fin de la portada de la portada
\maketitle

% Introducir aquí sus capítulos
% ------∨∨∨∨∨∨∨∨∨∨∨∨∨∨∨--------
\noindent\textbf{1. De las siguientes expresiones, identificar las proposiciones atomicas y los conectores lógicos. Traducir de lenguaje natural a lenguaje lógico:}\newpage
\chapter*{Ejercicio 2}
\section*{Ejercicio 2}

Para las siguientes parejas, escribir en lenguaje natural las fórmulas:

\[
p \land q, \quad p \lor q,\quad \neg p \land q, \quad p \land \neg q, \quad \neg p \lor q, \quad p \lor \neg q
\]

\subsection*{a) $p = 1 \text{ es primo}, \quad q = 1 \text{ es natural}$}

\begin{itemize}
    \item $p \land q$: 1 es primo y 1 es natural.
    \item $p \lor q$: 1 es primo o 1 es natural.
    \item $\neg p \land q$: No es cierto que 1 sea primo y 1 es natural.
    \item $p \land \neg q$: 1 es primo y no es cierto que 1 sea natural.
    \item $\neg p \lor q$: No es cierto que 1 sea primo o 1 es natural.
    \item $p \lor \neg q$: 1 es primo o no es cierto que 1 sea natural.
\end{itemize}

\subsection*{b) $p = \text{El gato no es un vegetal}, \quad q = \text{El perro es mamífero}$}

\begin{itemize}
    \item $p \land q$: El gato no es un vegetal y el perro es mamífero.
    \item $p \lor q$: El gato no es un vegetal o el perro es mamífero.
    \item $\neg p \land q$: El gato es un vegetal y el perro es un mamifero.
    \item $p \land \neg q$: El gato no es un vegetal y el perro no es mamífero.
    \item $\neg p \lor q$: El gato es un vegetal o el perro es mamífero.
    \item $p \lor \neg q$: El gato no es un vegetal o el perro no es mamífero.
\end{itemize}

\subsection*{c) $p = 5 < 7, \quad q = 3 \leq 10$}

\begin{itemize}
    \item $p \land q$: 5 es menor que 1 y 5 es menor que 10.
    \item $p \lor q$: 5 es menor que 1 o 5 es menor que 10.
    \item $\neg p \land q$: 5 no es menor que 1 y 5 es meno que 10.
    \item $p \land \neg q$: 5 es menor que 1 y 5 no es menor que 10.
    \item $\neg p \lor q$:  5 no es menor qe 1 o 5 es mayor que 10.
    \item $p \lor \neg q$: 5 es menor que 1 o 5 no es menor que 10.
\end{itemize}

 \newpage
\chapter*{Ejercicio 3}
% \section*{Ejercicio 3}

A partir de la siguiente gramática para expresiones proposicionales:

\[
E \to T \mid \neg E \mid E \land E \mid E \lor E \mid E \to E \mid (E)
\]
\[
T \to p \mid q \mid r \mid a \mid b \mid c \mid d
\]

\subsection*{a) $p \to q$}

\vspace*{\fill}
\begin{center}
\[
\begin{array}{c}
     \rightarrow \\
    / \ \backslash \\
   p   \quad  q 
\end{array}
\]
\end{center}
\vspace*{\fill}

\subsection*{b) $\neg(p \lor q)$}

\vspace*{\fill}
\begin{center}
\[
\begin{array}{c}
      \neg \\
      | \\
      \lor \\
     / \ \backslash \\
    p   \quad  q 
\end{array}
\]
\end{center}
\vspace*{\fill}

\subsection*{c) $(a \land b) \lor c \to (a \land d)$}

\vspace*{\fill}
\begin{center}
\[
\begin{array}{c}
      \rightarrow \\
      / \quad\ \quad \backslash \\
      \lor   \quad \quad \quad \land  \\
    / \ \backslash \quad\quad \quad / \ \backslash \\
   \land   \quad c \quad \quad   a \quad b\\
  / \ \backslash \quad \quad \quad \quad  \quad\\
 a   \quad  b  \quad \quad \quad\quad \quad
\end{array}
\]
\end{center}
\vspace*{\fill}

\subsection*{d) $(p \to a) \to (a \lor \neg b)$}

\vspace*{\fill}
\begin{center}
\[
\begin{array}{c}
       \rightarrow \\
      / \quad \ \backslash \\
     \rightarrow   \quad  \lor \\
    / \ \backslash \quad / \ \backslash\\
   p   \quad  a   \quad a \quad \neg\\
               \quad   | \\
                \quad   b
\end{array}
\]
\end{center}
\vspace*{\fill}

\subsection*{e) $\neg p \land \neg q \lor r$}

\vspace*{\fill}
\begin{center}
\[
\begin{array}{c}
      \lor \\
     / \ \backslash \\
    \land   \quad  r \\
   /  \ \backslash \quad \\
  \neg  \quad  \neg \quad \\
  | \quad \quad| \quad\\
  p \quad  \quad q \quad
\end{array}
\]
\end{center}
\vspace*{\fill}

\subsection*{f) $\neg a \to (b \land \neg c) \leftrightarrow \neg d$}

\vspace*{\fill}
\begin{center}
\[
\begin{array}{c}
      \leftrightarrow \\
     / \quad \ \backslash \\
    \to   \quad\quad  \neg \quad \\
   / \quad\ \backslash \quad \quad | \quad\\
   \neg    \quad \quad \land \quad d \quad \quad\\
      |\quad \quad       / \ \backslash \quad \quad \quad\\
      a\quad  \quad    b   \quad \neg  \quad \quad \\
      | \quad\\
      c \quad
\end{array}
\]
\end{center}
\vspace*{\fill}

\vspace*{\fill}\newpage
\chapter*{Ejercicio 4}
% \section*{Ejercicio 4}

\textbf{4.-} Utilizar el algoritmo de analisis de proposiciones sobre las siguientes fórmulas. Dibuja los arboles binarios resultantes. Señalar si el algoritmo acepta o no la fórmula: \newline 
\begin{center}
\begin{enumerate}
\renewcommand{\theenumi}{\alph{enumi}} %Letras minúsculas 
    \item $(p \land q)\lor r) \rightarrow ( p \land s)$
    \item $(p \rightarrow q) \neg \land r$
    \item $ \neg (p \rightarrow q) \rightarrow (q \lor \neg r)$
    \item $\neg p \land \neg q \lor r$
    \item $(\neg a \rightarrow (b \land \neg c)) \leftrightarrow \neg d $
\end{enumerate}
\end{center}

%Primero recordemos el algoritmo ANALYSIS(E): (Pseudocódigo)\\
%\newline

\textbf{Para a)}
\[
((p \land q)\lor r) \rightarrow ( p \land s)
\]  \newline
El algoritmo si acepta la fórmula y genera el siguiente árbol binario: \newline

\begin{center}
\[
\begin{array}{c}
\rightarrow \\
/ \quad \ \backslash \\
\lor \quad  \quad  \land \\
/ \ \backslash \quad / \ \backslash \\
\land \quad r \quad p \quad s \\
/ \ \backslash \quad \quad \quad \quad \quad\\
p \quad q \quad \quad \quad \quad \quad  \\
\end{array}
\]
\end{center}


\textbf{Para b)}
\[
(p \rightarrow q) \neg \land r
\]  \newline
El algoritmo NO acepta la fórmula y  NO genera el
árbol binario esto por no ser $wff$ (well formed formula).\\
\newline
El algoritmo regresa fail, puesto que al evaluar rankL(E) lo que recibe es: \\
$ (p \rightarrow q) \neg$ lo cual es ínvalido puesto que al tener la negación el algoritmo requiere el caso de tipo $\neg (E)$ lo cual no se tiene en este caso.\\
Regresa $Fail$.\\
\newline

\textbf{Para c)}
\[
\neg (p \rightarrow q) \rightarrow (q \lor \neg r)
\]  \newline
El algoritmo si acepta la fórmula y genera el siguiente árbol binario: \newline

\begin{center}
\[
\begin{array}{c}
\rightarrow \\
/ \quad \backslash \\
\neg \quad \quad \lor \\
\quad | \quad \quad / \quad \backslash \\
\quad \rightarrow \quad  q \quad \neg \\
/ \quad \backslash \quad \quad \quad | \\
p \quad \quad q \quad \quad \quad r\\
\end{array}
\]
\end{center}

\textbf{Para d)}
\[
\neg p \land \neg q \lor r
\]  \newline
El algoritmo si acepta la fórmula y genera el siguiente árbol binario: \newline

\begin{center}
\[
\begin{array}{c}
\land \\
/ \quad \backslash \\
\neg \quad \quad \lor \\
\quad | \quad \quad / \quad \backslash \\
\quad p \quad \quad  \neg \quad r \\
\quad \quad \quad | \quad \\
\quad \quad \quad q \quad \\
\end{array}
\]
\end{center}

\textbf{Para e)}
\[
(\neg a \rightarrow (b \land \neg c)) \leftrightarrow \neg d
\]  \newline
El algoritmo si acepta la fórmula y genera el siguiente árbol binario: \newline

\begin{center}
\[
\begin{array}{c}
\leftrightarrow \\
/ \quad \backslash \\
\rightarrow \quad \quad \neg \\
/ \quad \backslash \quad \quad | \\
\neg \quad \land \quad \quad d \\
| \quad / \quad \backslash \quad \quad  \\
a \quad b \quad \neg \quad \quad \\
\quad | \\
\quad c \\
\end{array}
\]
\end{center}\newpage
\chapter*{Ejercicio 5}
% \section*{Ejercicio 5}

\textbf{5.-} Demostrar que el algoritmo de analisis ANALYSIS(E) de expresiones proposicionales es completo, cuando la expresion tiene longitud finita. \\
\newline
\begin{center}
    Demostracion sobre los pasos del algoritmo
\end{center}

Despúes de haver visto el pseudocódigo del algoritmo, veremos el algoritmo en casos: \\
\begin{align*}
ANALYSIS(E) = \begin{cases}
    Tree (void,E,void)  & \text{si } E=\text{prop. atómica}. \\
    Tree (void,\neg, ANALISIS(E) & \text{si } E=\neg E \\
    Tree (ANALYSIS(rankL(E)),\diamondsuit,ANALYSIS(rankR(E)) & \text{si } E=E \diamondsuit E \\
\end{cases}
\end{align*}
\newline
\textbf{1) CASO BASE:}\\
\newline
Para E = prop. atómica (var p,q... o constante $\bot$ o $\top$)
Regresa $tree (void,E,void)$, si se cumple el algoritmo el regresa el árbol de E. \\
\newline

\textbf{2) HIPOTESIS DE INDUCCIÓN:}\\
\newline
Sean A y B proposiciones con las longitudes de A y B menores que $n$\\
\begin{itemize}
    \item ANALYSIS($A$) regresa el árbol sintáctico de A (se cumple para A).
    \item ANALYSIS($B$) regresa el árbol sintáctico de B (se cumple para B).
    %\item ANALYSIS($\neg A$) regresa el árbol sintáctico de $\neg A$ (se cumple para la negación).
    %\item ANALYSIS($A \diamondsuit B$) regresa el árbol sintáctico de $A \diamondsuit B$ (se cumple para operadores binarios).
\end{itemize}

\textbf{3) PASO INDUCTIVO:}\\
\newline
Por Demostrar para: ANALYSIS(E) es completo cuando E es de longitud $n$. \\
Tenemos dos casos, el caso de la negación y el caso con un operador binario:\\
\begin{itemize}
    \item Si $E=\neg A$, en esta situación caemos en el segundo caso de nuestro algoritmo\\
    Por lo tanto se debe aplicar $\neg ANALYSIS(A)$, y por \textbf{H.I.} sabemos que $ANALYSIS(A)$ se cumple. Así podemos decir que el algoritmo se cumple también para este caso.
    \item Si $E = A \diamondsuit B$, en esta situación se presenta el tercer caso definido por nuestro algoritmo\\
    Por lo tanto como se generará, \\ $Tree(ANALYSIS(rankL(E)),\diamondsuit,Tree(ANALYSIS(rankR(E)))$, por la función MainOP sabemos que $rankL(E)=A$ y $rankR(E)=B$, así tenemos $Tree(ANALYSIS(A),\diamondsuit,Tree(ANALYSIS(B))$, por \textbf{H.I.} sabemos que $ANALYSIS(A)$ y $ANALYSIS(B)$ se cumplen. Así podemos decir que el algoritmo también se cumple para este caso.
\end{itemize}
Asi vemos como se funciona para E cuando E es compuesta de longitud $n$. \\
\newline 
Además se E es $wff$ siempre llegaremos a una E atómica y si E NO es $wff$ el algoritmo regresará $fail$. \\
\newline
\textbf{Por lo tanto}, Demostramos que el algoritmo de ANALYSIS(E) es completo cuando la expresión tiene longitud finita.\newpage
\chapter*{Ejercicio 6}
\section*{Ejercicio 6}

Elaborar las tablas de verdad para las siguientes propocisoiones:
\[
a) \quad \neg (p \land q), \quad b)\quad \neg (p \lor q),\quad c) \quad (r \lor ( p \land q)) \to r,\] 
\[\quad d) \quad (p \land ( r \land q) ) \to  q, \quad e) \quad (p \to q) \leftrightarrow ( p \to r)
\] \\

\textit{a) }
% Tabla (a)
\[
\begin{array}{|c|c|c|c|c|}
\hline
p & q & p \land q & \neg (p \land q) \\
\hline
\text{V} & \text{V} & \text{V} & \text{F} \\
\hline
\text{V} & \text{F} & \text{F} & \text{V} \\
\hline
\text{F} & \text{V} & \text{F} & \text{V} \\
\hline
\text{F} & \text{F} & \text{V} & \text{V} \\
\hline
\end{array}
\]  


\textit{b) }
% Tabla (b)
\[
\begin{array}{|c|c|c|c|}
\hline
p & q & p \lor q & \neg (p \lor q) \\
\hline
\text{V} & \text{V} & \text{V} & \text{F} \\
\hline
\text{V} & \text{F} & \text{F} & \text{F} \\
\hline
\text{F} & \text{V} & \text{F} & \text{F} \\
\hline
\text{F} & \text{F} & \text{F} & \text{V} \\
\hline
\end{array}
\]

\textit{c) }
% Tabla (c)
\[
\begin{array}{|c|c|c|c|c|}
\hline
p & q & r & r \lor (p \land q) & (r \lor (p \land q)) \rightarrow r \\
\hline
\text{V} & \text{V} & \text{V} & \text{V} & \text{V} \\
\hline
\text{V} & \text{V} & \text{F} & \text{V} & \text{F} \\
\hline
\text{V} & \text{F} & \text{V} & \text{V} & \text{V} \\
\hline
\text{V} & \text{F} & \text{F} & \text{F} & \text{V} \\
\hline
\text{F} & \text{V} & \text{V} & \text{V} & \text{V} \\
\hline
\text{F} & \text{V} & \text{F} & \text{F} & \text{V} \\
\hline
\text{F} & \text{F} & \text{V} & \text{V} & \text{V} \\
\hline
\text{F} & \text{F} & \text{F} & \text{F} & \text{V} \\
\hline
\end{array}
\]

\textit{d) }

%Tabla(d)
\[
\begin{array}{|c|c|c|c|c|c|c|}
\hline
p & q & r & r  \land q & p \land (r \land q) & (p \land (r \land q )) \to q\\
\hline
\text{V} & \text{V} & \text{V} & \text{V}&\text{V} & \text{V} \\
\hline
\text{V} & \text{V} & \text{F} & \text{F} & \text{F} & \text{V}\\
\hline
\text{V} & \text{F} & \text{V} & \text{F} & \text{F} & \text{V} \\
\hline
\text{V} & \text{F} & \text{F} & \text{F} & \text{F} & \text{V}\\
\hline
\text{F} & \text{V} & \text{V} & \text{V} & \text{F} & \text{V}\\
\hline
\text{F} & \text{V} & \text{F} & \text{F} & \text{F} & \text{V}\\
\hline
\text{F} & \text{F} & \text{V} & \text{F} & \text{F} & \text{V}\\
\hline
\text{F} & \text{F} & \text{F} & \text{F} & \text{F} & \text{V}\\
\hline
\end{array}
\]

\textit{e) }
%Tabla (e)
\[
\begin{array}{|c|c|c|c|c|c|}
\hline
p & q & r & p \rightarrow q & p \rightarrow r & (p \rightarrow q) \leftrightarrow (p \rightarrow r) \\
\hline
\text{V} & \text{V} & \text{V} & \text{V} & \text{V} & \text{V} \\
\hline
\text{V} & \text{V} & \text{F} & \text{V} & \text{F} & \text{F} \\
\hline
\text{V} & \text{F} & \text{V} & \text{F} & \text{V} & \text{F} \\
\hline
\text{V} & \text{F} & \text{F} & \text{F} & \text{F} & \text{V} \\
\hline
\text{F} & \text{V} & \text{V} & \text{V} & \text{V} & \text{V} \\
\hline
\text{F} & \text{V} & \text{F} & \text{V} & \text{V} & \text{V} \\
\hline
\text{F} & \text{F} & \text{V} & \text{V} & \text{V} & \text{V} \\
\hline
\text{F} & \text{F} & \text{F} & \text{V} & \text{V} & \text{V} \\
\hline
\end{array}
\]\newpage
\textbf{7. Demuestra que la función del complemento regresa la negación de la fórmula.}

Esto es, que $comp(E)=\neg E$\\
\textbf{Proposición.} Sea $comp$ la siguiente función recursiva:
% \begin{alignat*}{2}
% 	comp(\top)      & = \bot                  & \quad & (i)   \\
% 	comp(\bot)      & = \top                  & \quad & (ii)  \\
% 	comp(p)         & = \neg p                & \quad & (iii) \\
% 	comp(\neg Q)    & = \neg comp(Q)          & \quad & (iv)  \\
% 	comp(P \land Q) & = comp(P) \land comp(Q) & \quad & (v)   \\
% 	comp(P \lor Q)  & = comp(P)\lor comp(Q)   & \quad & (vi)  \\
% \end{alignat*}
\begin{enumerate}
	\item $comp(\top) = \bot,\;comp(\bot) = \top,\;comp(p) = \neg p$ son atómicas.
	\item Si P y Q son fórmulas: $comp(\neg Q) = \neg comp(Q),\;comp(P \land Q) = comp(P) \land comp(Q),\;comp(P \lor Q) = comp(P)\lor comp(Q)$
\end{enumerate}
\indent Entonces se cumple que $comp(E)=\neg E$
\noindent\\
\textbf{Demostración:} Por inducción estructural sobre las fórmulas.\\
\indent
\textbf{Caos base.} Cuando $E$ es atómica tal que $E=p$ donde $p$ es una proposición ó $E=\top$ ó $E=\bot$
% \vspace{-20px}
\begin{multicols}{3}
	\begin{alignat*}{2}
		E=p\therefore                               \\
		comp(E) & = comp(p)                         \\
		        & =\neg p   & \quad & \text{Por}(1) \\
	\end{alignat*}

	\begin{alignat*}{2}
		E=\top\therefore                               \\
		comp(E) & = comp(\top)                         \\
		        & =\bot        & \quad & \text{Por}(1) \\
	\end{alignat*}

	\begin{alignat*}{2}
		E=\bot\therefore                               \\
		comp(E) & = comp(\bot)                         \\
		        & =\top        & \quad & \text{Por}(1) \\
	\end{alignat*}
\end{multicols}

\textbf{Hipótesis de inducción:} Supongamos que $comp(P)=\neg P$ y $comp(Q)=\neg Q$\\
\textbf{Paso inductivo:}\newpage
\textbf{8. Demostra que a partir de los conjuntos de proposiciones dados $\Gamma$, si las siguientes proposiciones son o no consecuencias lógicas utilizando interpretaciones.}
\begin{multicols}{2}
	\begin{enumerate}[label=\alph*)]
		\item $\Gamma = \{p\land q, r\lor q\}$, proposición: $p \land q\lor r$
		\item $\Gamma = \{p\leftrightarrow q,p\rightarrow \neg r,r\rightarrow s\}$, proposición: $q\rightarrow s$
		\item $\Gamma = \{p\leftrightarrow q,p\rightarrow \neg r,r\rightarrow s\}$, proposición: $\neg (p\land r)$
		\item $\Gamma = \{p\lor q, q\rightarrow r, \neg r \lor s\}$, proposición: $(p\lor q)\rightarrow s$
		\item $\Gamma = \{p\land q, q\rightarrow r, r \lor \neg s\}$, proposición: $(p \land q)\rightarrow r$
	\end{enumerate}
\end{multicols}

\textbf{Mostrar que a) 	$\Gamma=\{p\land q, r\lor q\} \vDash p \land q\lor r$.}\\
Suponemos la veracidad de $\mathcal{I}(\Gamma)=1$\\
Sea $\mathcal{I}$ un modelo $\Gamma$. Tenemos que demostrar que $\mathcal{I}((p \land q)\lor r)=1$.\\
Como $\mathcal{I}(p\lor q)=1$, entonces $\mathcal{I}(p)=1=\mathcal{I}(q)$ y para $\mathcal{I}(r\lor q)$ tenemos dos casos\\
\indent i) Cuando $\mathcal{I}(r)=1$, y como $\mathcal{I}(q)=1$ entonces $\mathcal{I}(q\lor r)=1$ siempre, por lo que $\mathcal{I}(p \land q\lor r)=1$ dodo que $\mathcal{I}(p \land q)=1$ y $\mathcal{I}(r)=1$\\
\indent ii) Por otro lado, Cuando $\mathcal{I}(r)=0$, como $\mathcal{I}(q)=1$, entonces $\mathcal{I}(q\lor r)=1$\\
$\therefore \mathcal{I}((p\land q)\lor r)=1$\\
\indent
$\therefore$ se concluye que es onsecuencia lógica. $\blacksquare$
\vspace{10px}

\textbf{Mostrar que b)} $\Gamma = \{p\leftrightarrow q,p\rightarrow \neg r,r\rightarrow s\} \vDash q\rightarrow s$\\
Suponemos la veracidad de $\mathcal{I}(\Gamma)=1$\\
Tenemos dos casos:\\
 \indent i) Si $\mathcal{I}(q)=0$ entonces $\mathcal{I}(q\rightarrow s)=1$ por lo que es trivial.\\
\indent ii) Si $\mathcal{I}(q)=1$, entonces $\mathcal{I}(p)=1$ para que sea $\mathcal{I}(p\leftrightarrow q)=1$, por lo que $\mathcal{I}(\neg r)=1$ necesariamente, pues $\mathcal{I}(p\rightarrow \neg r)=1$, entonces $\mathcal{I}(r)=0$, quiere decir que $\mathcal{I}(r\rightarrow s)=1$, en particular para $\mathcal{I}(s)=0$, por lo que, si $\mathcal{I}(q)=1$, como lo definimos anteriormente y si $\mathcal{I}(s)=0$, quiere decir que $\mathcal{I}(r\rightarrow s)=0$\\
$\therefore$ No es consecuencia lógica $\blacksquare$
\vspace{10px}

\textbf{Mostrar que c) $\Gamma = \{p\leftrightarrow q,p\rightarrow \neg r,r\rightarrow s\} \vDash \neg (p\land r)$}\\
Suponemos la veracidad de $\mathcal{I}(\Gamma)=1$\\
Tenemos dos casos:\\
\indent i) Si $\mathcal{I}(p)=0$, entonces $\mathcal{I}(\neg(p\land r))=1$ pues $\mathcal{I}(p\land r)=0$.\\
\indent ii) Si $\mathcal{I}(p)=1$ como $\mathcal{I}(p \leftrightarrow q)=1$ entonces $\mathcal{I}(q)=1$, esto quiere decir que, como $\mathcal{I}(q\rightarrow \neg r)=1$, tiene que pasar que $\mathcal{I}(\neg r)=1$, por lo que $\mathcal{I}(r)=0$.\\
Esto quiere decir que $\mathcal{I}(p\land r)=0$ y $\mathcal{I}(\neg (p\land r))=1$\\
$\therefore$ Si es consecuencia lógica.$\blacksquare$
\vspace{10px}

\textbf{Mostrar que d) $\Gamma = \{p\lor q, q\rightarrow r, \neg r \lor s\}\vDash(p\lor q)\rightarrow s$}\\
Suponemos la veracidad de $\mathcal{I}(\Gamma)=1$\\
Tenemos dos casos:\\
\indent i)Supongamos que $\mathcal{I}(q)=1$, dado que $\mathcal{I}(p\rightarrow r)=1$ quiere decir que $\mathcal{I}(r)=1$, entonces $\mathcal{I}(\neg r)=0$, y como $\mathcal{I}(\neg r\lor s)=1$ tiene que pasar que $\mathcal{I}(s)=1$, dado que suponemos que $\mathcal{I}(p\lor q)=1$ es necesario que $\mathcal{I}(p)=1$ pues $\mathcal{I}(q)=0$ como suposimos anteriormente. $\therefore\mathcal{I}((p\lor q)\rightarrow s)=1$\\
\indent ii)Supongamos$\mathcal{I}(q)=0$, entonces, en particular, suponemos que $\mathcal{I}(r)=0$, esto significa que $\mathcal{I}(\neg r)=1$, como $\mathcal{I}(\neg r \lor s)=1$ puede pasar que $\mathcal{I}(s)=0$, y dado que $\mathcal{I}(p\lor q)=1$ tiene que pasar que $\mathcal{I}(p)=1$ entonces decimos que $\mathcal{I}((p\lor q)\rightarrow s)=0$ puesto que $\mathcal{I}(p\lor q)=1$ pero $\mathcal{I}(s)=0$\\
$\therefore$ No es consecuencia lógica. $\blacksquare$
\vspace{10px}

\textbf{Mostrar que e) $\Gamma = \{p\land q, q\rightarrow r, r \lor \neg s\}\vDash (p \land q)\rightarrow r$}\\
Suponemos la veracidad de $\mathcal{I}(\Gamma)=1$\\
Dado que $\mathcal{I}(p\land q)=1$ tiene que pasar que $\mathcal{I}(p)=1=\mathcal{I}(q)$, entonces es necesario que $\mathcal{I}(r)=1$ pues $\mathcal{I}(q\rightarrow r)=1$, quiere decir se cumple 
$\mathcal{I}(r \lor \neg s)=1$ pues basta que al menos uno sea $1$ para que la proposición se cumpla, lo que quiere decir que $\mathcal{I}((p\land q)\rightarrow r)=1$\\
$\therefore$ Es consecuencia lógica. $\blacksquare$\newpage
\chapter*{Ejercicio 10}
% \section*{Ejercicio 10}

\textbf{10.} Probar que el operador de disyuncion exclusiva (XOR) $p \veebar q$ es equivalente a $\neg (p \land q) \land (p \lor q)$.\\
\newline
(Proposiciones equivalentes). Decimos que dos formulas proposicionales $p$ y $q$ son equivalentes si y solo si en todos sus posibles estados tienen el mismo valor de verdad.
Por lo tanto para probar la equivalencia de $p \veebar q$ y $\neg (p \land q) \land (p \lor q)$ generaremos su tabla de verdad. \\
\newline
\[
\begin{array}{|c|c|c|c|c|c|c|}
\hline
p & q & p \land q & p \lor q & \neg (p \land a) & \neg (p \land q) \land (p \lor q) & p \veebar q \\
\hline
0 & 0 & 0 & 0 & 1 & 0 & 0 \\
\hline
0 & 1 & 0 & 1 & 1 & 1 & 1 \\
\hline
1 & 0 & 0 & 1 & 1 & 1 & 1 \\
\hline
1 & 1 & 1 & 1 & 0 & 0 & 0 \\
\hline
\end{array}
\]

Dado este resultado podemos ver que los valores de $\neg (p \land q) \land (p \lor q)$ y $p \veebar q$ son iguales en todos sus posibles estados. Por lo tanto son proposiciones equivalentes.
\newpage
\chapter*{Ejercicio 11}
\section*{Ejercicio 11}

Demostrar que dada una fórmula de lógica proposicional $E$, la altura es menor o igual que la longitud. Esto es $h(E) \leq \text{len}(E)$. \\

\begin{enumerate}
    \item[1)] Caso base: consideramos que $\text{len}(E) = 1$ y altura $h(E) = 1$, entonces este caso hace que se cumpla que
    \[
    h(E) \leq \text{len}(E)
    \]
    por lo tanto, es correcto.

    \item[2)] Hipótesis de inducción: suponemos que para $E_1$ y $E_2$ se cumple que
    \[
    h(E_1) \leq \text{len}(E_1) \quad \text{y} \quad h(E_2) \leq \text{len}(E_2).
    \]
    Queremos probar que para las fórmulas $E = E_1  \diamondsuit E_2$, o $\neg E_1$, también se cumple la desigualdad.

    \begin{itemize}
        \item Caso $E = E_1 \diamondsuit E_2$:
        \begin{itemize}
            \item[]  Longitud:
            \[
            \text{len}(E_1 \diamondsuit E_2) = \text{len}(E_1) + \text{len}(E_2) + 1
            \]
            \item[] Altura:
            \[
            h(E_1 \diamondsuit E_2) = 1 + \max(h(E_1), h(E_2))
            \]
            Por la hipótesis de inducción, tenemos que $h(E_1) \leq \text{len}(E_1)$ y $h(E_2) \leq \text{len}(E_2)$. Entonces:
            \[
            h(E_1 \land E_2) = 1 + \max(h(E_1), h(E_2)) \leq 1 + \max(\text{len}(E_1), \text{len}(E_2)) \leq \text{len}(E_1) + \text{len}(E_2) + 1
            \]
            \[
            = \text{len}(E_1 \land E_2)
            \]
            Por lo tanto, se cumple que $h(E_1 \land E_2) \leq \text{len}(E_1 \land E_2)$.
        \end{itemize}
        \item Caso $E = \neg E_1$:
        \begin{itemize}
            \item []Longitud:
            \[
            \text{len}(\neg E_1) = \text{len}(E_1) + 1
            \]
            \item[] Altura:
            \[
            h(\neg E_1) = h(E_1) + 1
            \]
            Por la hipótesis de inducción, sabemos que $h(E_1) \leq \text{len}(E_1)$, entonces se cumple que
            \[
            h(\neg E_1) = h(E_1) + 1 \leq \text{len}(E_1) + 1 = \text{len}(\neg E_1)
            \]
            Así, se cumple que $h(\neg E_1) \leq \text{len}(\neg E_1)$.
        \end{itemize}
    \end{itemize}
\end{enumerate}

Conclusión: por inducción, sabemos que para una fórmula proposicional $E$, se cumple que
\[
h(E) \leq \text{len}(E)
\]
\end{document}