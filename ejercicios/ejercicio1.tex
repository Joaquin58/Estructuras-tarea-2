\noindent\textbf{1. De las siguientes expresiones, identificar las proposiciones atomicas y los conectores lógicos. Traducir de lenguaje natural a lenguaje lógico:}

\begin{multicols}{2}
	\begin{enumerate}[label=\alph*)]
		\item Penélope es griega.
		\item Alonso Quijano no está cuerdo.
		\item Si Juan fue al cine, seguro que Lupe también.
		\item Melibea no está triste, porque cursó Estructuras Discretas.
		\item Juan come y bebe.
		\item Cuando María estudia, no reprueba los exámenes.
		\item Armin no fuma ni bebe.
		\item Juana juega fútbol, pero no baloncesto.
	\end{enumerate}
\end{multicols}

\textbf{a)}
\vspace{-10px}
\begin{multicols}{2}
    \noindent
    \begin{gather*}
        p = \text{ Penélope es griega}
    \end{gather*}
    \columnbreak
    \begin{gather*}
        p 
    \end{gather*}
\end{multicols}
\textbf{b)}
\vspace{-10px}
\begin{multicols}{2}
    \noindent
    \begin{align*}
        p &= \text{ Alonso Quijano está cuerdo}\\
    \end{align*}
    \columnbreak
    \begin{align*}
        \neg p
    \end{align*}
\end{multicols}
\textbf{c)}
\vspace{-10px}
\begin{multicols}{2}
    \noindent
    \begin{align*}
        p &= \text{ Juan fue al cine}\\
        q &= \text{ Lupe fue al cine}
    \end{align*}
    \columnbreak
    \begin{align*}
        p\implies q
    \end{align*}
\end{multicols}
\textbf{d)}
\vspace{-10px}
\begin{multicols}{2}
    \noindent
    \begin{align*}
        p &= \text{ Melibea cursó Estructuras Discretas}\\
        q &= \text{ Melibea está triste}
    \end{align*}
    \columnbreak
    \begin{align*}
        p\implies \neg q
    \end{align*}
\end{multicols}
\textbf{e)}
\vspace{-10px}
\begin{multicols}{2}
    \noindent
    \begin{align*}
        p &= \text{ Juan come}\\
        q &= \text{ Juan bebe}
    \end{align*}
    \columnbreak
    \begin{align*}
        p\land  q
    \end{align*}
\end{multicols}
\textbf{f)}
\vspace{-10px}
\begin{multicols}{2}
    \noindent
    \begin{align*}
        p &= \text{ María estudia}\\
        q &= \text{ Maria reprueba los exámenes}
    \end{align*}
    \columnbreak
    \begin{align*}
        p\implies \neg q
    \end{align*}
\end{multicols}
\textbf{g)}
\vspace{-10px}
\begin{multicols}{2}
    \noindent
    \begin{align*}
        p &= \text{ Armin fuma}\\
        q &= \text{ Armin bebe}
    \end{align*}
    \columnbreak
    \begin{align*}
        \neg p\land \neg q
    \end{align*}
\end{multicols}
\textbf{h)}
\vspace{-10px}
\begin{multicols}{2}
    \noindent
    \begin{align*}
        p &= \text{ Juana juega fútbol}\\
        q &= \text{ Juana juega baloncesto}
    \end{align*}
    \columnbreak
    \begin{align*}
        p\land \neg q
    \end{align*}
\end{multicols}

