\textbf{7. Demuestra que la función del complemento regresa la negación de la fórmula.}

Esto es, que $comp(E)=\neg E$\\
\textbf{Proposición.} Sea $comp$ la siguiente función recursiva:
% \begin{alignat*}{2}
% 	comp(\top)      & = \bot                  & \quad & (i)   \\
% 	comp(\bot)      & = \top                  & \quad & (ii)  \\
% 	comp(p)         & = \neg p                & \quad & (iii) \\
% 	comp(\neg Q)    & = \neg comp(Q)          & \quad & (iv)  \\
% 	comp(P \land Q) & = comp(P) \land comp(Q) & \quad & (v)   \\
% 	comp(P \lor Q)  & = comp(P)\lor comp(Q)   & \quad & (vi)  \\
% \end{alignat*}
\begin{enumerate}
	\item $comp(\top) = \bot,\;comp(\bot) = \top,\;comp(p) = \neg p$ son atómicas.
	\item Si P y Q son fórmulas: $comp(\neg Q) = \neg comp(Q),\;comp(P \land Q) = comp(P) \land comp(Q),\;comp(P \lor Q) = comp(P)\lor comp(Q)$
\end{enumerate}
\indent Entonces se cumple que $comp(E)=\neg E$
\noindent\\
\textbf{Demostración:} Por inducción estructural sobre las fórmulas.\\
\indent
\textbf{Caos base.} Cuando $E$ es atómica tal que $E=p$ donde $p$ es una proposición ó $E=\top$ ó $E=\bot$
\begin{multicols}{3}
	\begin{alignat*}{2}
		E       & = p \text{ tal que } & \quad & \text{p es atómica}  \\
		comp(E) & = comp(p)                                           \\
		        & = \neg p             & \quad & \text{Por p atómica} \\
	\end{alignat*}

	\begin{alignat*}{2}
		E=\top\therefore                         \\
		comp(E) & = comp(\top)                   \\
		        & = \neg \top  & \quad & \text{} \\
		        & = \bot       & \quad & \text{} \\
	\end{alignat*}

	\begin{alignat*}{2}
		E=\bot\therefore                         \\
		comp(E) & = comp(\bot)                   \\
		        & =\neg \bot   & \quad & \text{} \\
		        & =\top        & \quad & \text{} \\
	\end{alignat*}
\end{multicols}

\textbf{Hipótesis de inducción:} Supongamos que se cumple para dos proposiciones $P$, $Q$ tales que  $comp(P)=\neg P$ y $comp(Q)=\neg Q$\\
\textbf{Paso inductivo: Por demostrar que se cumple para los pasos recurisvos 
de la función $comp(E)=\neg E$}
\begin{multicols}{3}
	\noindent
	\begin{align*}
		comp(\neg Q)              & = \neg comp(Q) \\
		\text{Por H.I}            & = \neg \neg Q  \\
		\text{Por doble negación} & =  Q           \\
	\end{align*}
\noindent
	\begin{align*}
		comp(P\land Q)      & = comp(P)\land comp(Q) \\
		\text{Por H.I}      & = \neg P \land \neg Q  \\
		\text{Por deMorgan} & = \neg (P\lor Q)       \\
	\end{align*}
\noindent
	\begin{align*}
		comp(P\lor Q)       & = \neg comp(P) \lor \neg comp(Q) \\
		\text{Por H.I}      & = \neg P \lor \neg Q             \\
		\text{Por deMorgan} & = \neg (P\land Q)                \\
	\end{align*}
\end{multicols}

$\therefore$ Se concluye que se cumple para todos los casos recurisvos de la función del complemento se cumple que$$comp(E)=\neg E\text{, para cualquier fórmula}$$