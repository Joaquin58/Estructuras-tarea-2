\chapter*{Ejercicio 5}
% \section*{Ejercicio 5}

\textbf{5.-} Demostrar que el algoritmo de analisis ANALYSIS(E) de expresiones proposicionales es completo, cuando la expresion tiene longitud finita. \\
\newline
\begin{center}
    Demostracion sobre los pasos del algoritmo
\end{center}

Despúes de haver visto el pseudocódigo del algoritmo, veremos el algoritmo en casos: \\
\begin{align*}
ANALYSIS(E) = \begin{cases}
    Tree (void,E,void)  & \text{si } E=\text{prop. atómica}. \\
    Tree (void,\neg, ANALISIS(E) & \text{si } E=\neg E \\
    Tree (ANALYSIS(rankL(E)),\diamondsuit,ANALYSIS(rankR(E)) & \text{si } E=E \diamondsuit E \\
\end{cases}
\end{align*}
\newline
\textbf{1) CASO BASE:}\\
\newline
Para E = prop. atómica (var p,q... o constante $\bot$ o $\top$)
Regresa $tree (void,E,void)$, si se cumple el algoritmo el regresa el árbol de E. \\
\newline

\textbf{2) HIPOTESIS DE INDUCCIÓN:}\\
\newline
Sean A y B proposiciones con las longitudes de A y B menores que $n$\\
\begin{itemize}
    \item ANALYSIS($A$) regresa el árbol sintáctico de A (se cumple para A).
    \item ANALYSIS($B$) regresa el árbol sintáctico de B (se cumple para B).
    %\item ANALYSIS($\neg A$) regresa el árbol sintáctico de $\neg A$ (se cumple para la negación).
    %\item ANALYSIS($A \diamondsuit B$) regresa el árbol sintáctico de $A \diamondsuit B$ (se cumple para operadores binarios).
\end{itemize}

\textbf{3) PASO INDUCTIVO:}\\
\newline
Por Demostrar para: ANALYSIS(E) es completo cuando E es de longitud $n$. \\
Tenemos dos casos, el caso de la negación y el caso con un operador binario:\\
\begin{itemize}
    \item Si $E=\neg A$, en esta situación caemos en el segundo caso de nuestro algoritmo\\
    Por lo tanto se debe aplicar $\neg ANALYSIS(A)$, y por \textbf{H.I.} sabemos que $ANALYSIS(A)$ se cumple. Así podemos decir que el algoritmo se cumple también para este caso.
    \item Si $E = A \diamondsuit B$, en esta situación se presenta el tercer caso definido por nuestro algoritmo\\
    Por lo tanto como se generará, \\ $Tree(ANALYSIS(rankL(E)),\diamondsuit,Tree(ANALYSIS(rankR(E)))$, por la función MainOP sabemos que $rankL(E)=A$ y $rankR(E)=B$, así tenemos $Tree(ANALYSIS(A),\diamondsuit,Tree(ANALYSIS(B))$, por \textbf{H.I.} sabemos que $ANALYSIS(A)$ y $ANALYSIS(B)$ se cumplen. Así podemos decir que el algoritmo también se cumple para este caso.
\end{itemize}
Asi vemos como se funciona para E cuando E es compuesta de longitud $n$. \\
\newline 
Además se E es $wff$ siempre llegaremos a una E atómica y si E NO es $wff$ el algoritmo regresará $fail$. \\
\newline
\textbf{Por lo tanto}, Demostramos que el algoritmo de ANALYSIS(E) es completo cuando la expresión tiene longitud finita.