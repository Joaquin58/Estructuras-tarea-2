\chapter*{Ejercicio 4}
% \section*{Ejercicio 4}

\textbf{4.-} Utilizar el algoritmo de analisis de proposiciones sobre las siguientes fórmulas. Dibuja los arboles binarios resultantes. Señalar si el algoritmo acepta o no la fórmula: \newline 
\begin{center}
\begin{enumerate}
\renewcommand{\theenumi}{\alph{enumi}} %Letras minúsculas 
    \item $(p \land q)\lor r) \rightarrow ( p \land s)$
    \item $(p \rightarrow q) \neg \land r$
    \item $ \neg (p \rightarrow q) \rightarrow (q \lor \neg r)$
    \item $\neg p \land \neg q \lor r$
    \item $(\neg a \rightarrow (b \land \neg c)) \leftrightarrow \neg d $
\end{enumerate}
\end{center}

%Primero recordemos el algoritmo ANALYSIS(E): (Pseudocódigo)\\
%\newline

\textbf{Para a)}
\[
((p \land q)\lor r) \rightarrow ( p \land s)
\]  \newline
El algoritmo si acepta la fórmula y genera el siguiente árbol binario: \newline

\begin{center}
\[
\begin{array}{c}
\rightarrow \\
/ \quad \ \backslash \\
\lor \quad  \quad  \land \\
/ \ \backslash \quad / \ \backslash \\
\land \quad r \quad p \quad s \\
/ \ \backslash \quad \quad \quad \quad \quad\\
p \quad q \quad \quad \quad \quad \quad  \\
\end{array}
\]
\end{center}


\textbf{Para b)}
\[
(p \rightarrow q) \neg \land r
\]  \newline
El algoritmo NO acepta la fórmula y  NO genera el
árbol binario esto por no ser $wff$ (well formed formula).\\
\newline
El algoritmo regresa fail, puesto que al evaluar rankL(E) lo que recibe es: \\
$ (p \rightarrow q) \neg$ lo cual es ínvalido puesto que al tener la negación el algoritmo requiere el caso de tipo $\neg (E)$ lo cual no se tiene en este caso.\\
Regresa $Fail$.\\
\newline

\textbf{Para c)}
\[
\neg (p \rightarrow q) \rightarrow (q \lor \neg r)
\]  \newline
El algoritmo si acepta la fórmula y genera el siguiente árbol binario: \newline

\begin{center}
\[
\begin{array}{c}
\rightarrow \\
/ \quad \backslash \\
\neg \quad \quad \lor \\
\quad | \quad \quad / \quad \backslash \\
\quad \rightarrow \quad  q \quad \neg \\
/ \quad \backslash \quad \quad \quad | \\
p \quad \quad q \quad \quad \quad r\\
\end{array}
\]
\end{center}

\textbf{Para d)}
\[
\neg p \land \neg q \lor r
\]  \newline
El algoritmo si acepta la fórmula y genera el siguiente árbol binario: \newline

\begin{center}
\[
\begin{array}{c}
\land \\
/ \quad \backslash \\
\neg \quad \quad \lor \\
\quad | \quad \quad / \quad \backslash \\
\quad p \quad \quad  \neg \quad r \\
\quad \quad \quad | \quad \\
\quad \quad \quad q \quad \\
\end{array}
\]
\end{center}

\textbf{Para e)}
\[
(\neg a \rightarrow (b \land \neg c)) \leftrightarrow \neg d
\]  \newline
El algoritmo si acepta la fórmula y genera el siguiente árbol binario: \newline

\begin{center}
\[
\begin{array}{c}
\leftrightarrow \\
/ \quad \backslash \\
\rightarrow \quad \quad \neg \\
/ \quad \backslash \quad \quad | \\
\neg \quad \land \quad \quad d \\
| \quad / \quad \backslash \quad \quad  \\
a \quad b \quad \neg \quad \quad \\
\quad | \\
\quad c \\
\end{array}
\]
\end{center}