\chapter*{Ejercicio 11}
\section*{Ejercicio 11}

Demostrar que dada una fórmula de lógica proposicional $E$, la altura es menor o igual que la longitud. Esto es $h(E) \leq \text{len}(E)$. \\

\begin{enumerate}
    \item[1)] Caso base: consideramos que $\text{len}(E) = 1$ y altura $h(E) = 1$, entonces este caso hace que se cumpla que
    \[
    h(E) \leq \text{len}(E)
    \]
    por lo tanto, es correcto.

    \item[2)] Hipótesis de inducción: suponemos que para $E_1$ y $E_2$ se cumple que
    \[
    h(E_1) \leq \text{len}(E_1) \quad \text{y} \quad h(E_2) \leq \text{len}(E_2).
    \]
    Queremos probar que para las fórmulas $E = E_1  \diamondsuit E_2$, o $\neg E_1$, también se cumple la desigualdad.

    \begin{itemize}
        \item Caso $E = E_1 \diamondsuit E_2$:
        \begin{itemize}
            \item[]  Longitud:
            \[
            \text{len}(E_1 \diamondsuit E_2) = \text{len}(E_1) + \text{len}(E_2) + 1
            \]
            \item[] Altura:
            \[
            h(E_1 \diamondsuit E_2) = 1 + \max(h(E_1), h(E_2))
            \]
            Por la hipótesis de inducción, tenemos que $h(E_1) \leq \text{len}(E_1)$ y $h(E_2) \leq \text{len}(E_2)$. Entonces:
            \[
            h(E_1 \diamondsuit E_2) = 1 + \max(h(E_1), h(E_2)) \leq 1 + \max(\text{len}(E_1), \text{len}(E_2)) \leq \text{len}(E_1) + \text{len}(E_2) + 1
            \]
            \[
            = \text{len}(E_1 \diamondsuit E_2)
            \]
            Por lo tanto, se cumple que $h(E_1 \diamondsuit E_2) \leq \text{len}(E_1 \diamondsuit E_2)$.
        \end{itemize}
        \item Caso $E = \neg E_1$:
        \begin{itemize}
            \item []Longitud:
            \[
            \text{len}(\neg E_1) = \text{len}(E_1) + 1
            \]
            \item[] Altura:
            \[
            h(\neg E_1) = h(E_1) + 1
            \]
            Por la hipótesis de inducción, sabemos que $h(E_1) \leq \text{len}(E_1)$, entonces se cumple que
            \[
            h(\neg E_1) = h(E_1) + 1 \leq \text{len}(E_1) + 1 = \text{len}(\neg E_1)
            \]
            Así, se cumple que $h(\neg E_1) \leq \text{len}(\neg E_1)$.
        \end{itemize}
    \end{itemize}
\end{enumerate}

Conclusión: por inducción, sabemos que para una fórmula proposicional $E$, se cumple que
\[
h(E) \leq \text{len}(E)
\]