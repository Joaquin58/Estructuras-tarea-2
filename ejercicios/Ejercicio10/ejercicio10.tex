\chapter*{Ejercicio 10}
% \section*{Ejercicio 10}

\textbf{10.} Probar que el operador de disyuncion exclusiva (XOR) $p \veebar q$ es equivalente a $\neg (p \land q) \land (p \lor q)$.\\
\newline
(Proposiciones equivalentes). Decimos que dos formulas proposicionales $p$ y $q$ son equivalentes si y solo si en todos sus posibles estados tienen el mismo valor de verdad.
Por lo tanto para probar la equivalencia de $p \veebar q$ y $\neg (p \land q) \land (p \lor q)$ generaremos su tabla de verdad. \\
\newline
\[
\begin{array}{|c|c|c|c|c|c|c|}
\hline
p & q & p \land q & p \lor q & \neg (p \land a) & \neg (p \land q) \land (p \lor q) & p \veebar q \\
\hline
0 & 0 & 0 & 0 & 1 & 0 & 0 \\
\hline
0 & 1 & 0 & 1 & 1 & 1 & 1 \\
\hline
1 & 0 & 0 & 1 & 1 & 1 & 1 \\
\hline
1 & 1 & 1 & 1 & 0 & 0 & 0 \\
\hline
\end{array}
\]

Dado este resultado podemos ver que los valores de $\neg (p \land q) \land (p \lor q)$ y $p \veebar q$ son iguales en todos sus posibles estados. Por lo tanto son proposiciones equivalentes.
