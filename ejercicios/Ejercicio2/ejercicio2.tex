\chapter*{Ejercicio 2}
% \section*{Ejercicio 2}

Para las siguientes parejas, escribir en lenguaje natural las fórmulas:

\[
p \land q, \quad p \lor q,\quad \neg p \land q, \quad p \land \neg q, \quad \neg p \lor q, \quad p \lor \neg q
\]

\subsection*{a) $p = 1 \text{ es primo}, \quad q = 1 \text{ es natural}$}

\begin{itemize}
    \item $p \land q$: 1 es primo y 1 es natural.
    \item $p \lor q$: 1 es primo o 1 es natural.
    \item $\neg p \land q$: No es cierto que 1 sea primo y 1 es natural.
    \item $p \land \neg q$: 1 es primo y no es cierto que 1 sea natural.
    \item $\neg p \lor q$: No es cierto que 1 sea primo o 1 es natural.
    \item $p \lor \neg q$: 1 es primo o no es cierto que 1 sea natural.
\end{itemize}

\subsection*{b) $p = \text{El gato no es un vegetal}, \quad q = \text{El perro es mamífero}$}

\begin{itemize}
    \item $p \land q$: El gato no es un vegetal y el perro es mamífero.
    \item $p \lor q$: El gato no es un vegetal o el perro es mamífero.
    \item $\neg p \land q$: El gato es un vegetal y el perro es un mamifero.
    \item $p \land \neg q$: El gato no es un vegetal y el perro no es mamífero.
    \item $\neg p \lor q$: El gato es un vegetal o el perro es mamífero.
    \item $p \lor \neg q$: El gato no es un vegetal o el perro no es mamífero.
\end{itemize}

\subsection*{c) $p = 5 < 7, \quad q = 3 \leq 10$}

\begin{itemize}
    \item $p \land q$: 5 es menor que 1 y 5 es menor que 10.
    \item $p \lor q$: 5 es menor que 1 o 5 es menor que 10.
    \item $\neg p \land q$: 5 no es menor que 1 y 5 es meno que 10.
    \item $p \land \neg q$: 5 es menor que 1 y 5 no es menor que 10.
    \item $\neg p \lor q$:  5 no es menor qe 1 o 5 es mayor que 10.
    \item $p \lor \neg q$: 5 es menor que 1 o 5 no es menor que 10.
\end{itemize}

 